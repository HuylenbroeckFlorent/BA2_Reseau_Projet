\documentclass[]{article}
\usepackage[utf8]{inputenc}
\usepackage{fullpage}
\usepackage{gensymb}
\usepackage{graphicx}
\usepackage{amsmath}

\begin{document}

\title{
{\Huge Rapport de projet n\degree 1}\\
Réseau I\\
\author{Membres du groupe:\\
\textbf{SAHLI Yacine}\\
\textbf{HUYLENBROECK Florent}\\
}}

\date{Année Académique 2017-2018\\
Bachelier en Sciences Informatiques\\
\vspace{1cm}
Faculté des Sciences, Université de Mons}

\maketitle
%---------------------------------------------------------------------------------------------------------
\newpage
\section{Utilisation de l'application}
\subsection{Lancement de l'application}
Pour démarrer l'application, lancer le fichier Main.java du package \emph{reso.examples.gobackn}.
\subsection{Paramètres modifiables}
\begin{itemize}
\item \emph{reso.examples.gobackn.GoBackNReceiverApp.java}\\
\begin{itemize}
\item NUMBER\_ OF\_ EVENT et les variables commençant par PROB\_ permettent de modifier les probabilités que des évènements inattendus se produisent, selon la formule : \[\frac{\text{PROB\_ EVENT}}{\text{NUMBER\_ OF\_ EVENT}}\]\\
\emph{exemple} : Si l'on souhaite que $5\%$ des ACK ne soient pas envoyés,\\ PROB\_ ACK\_ NOT\_ SENT=5; et NUMBER\_ OF\_ EVENT=100;\\\\
La variable NUMBER\_ OF\_ EVENT sert principalement à augmenter/réduire toutes les probabilités en une seule fois.
\item SMALL\_ DELAY\_ RANGE\_ MIN et SMALL\_ DELAY\_ RANGE\_ MAX correspondent à l'intervalle dans lequel sera choisi les petits délais aléatoires avant d'envoyer certains ACK.\\
\end{itemize}
\item \emph{reso.examples.gobackn.GoBackNSenderApp.java}\\
\begin{itemize}
\item TIMEOUT\_ DELAY modifie le délai avant un timeout.
\item PACKET\_ SENT modifie le nombre de paquets à envoyer.
\end{itemize}
\end{itemize}
\subsection{Comportement de l'application}
Une fois démarrée, une série de log s'afficheront dans la console, détaillant étape par étape l'échange de paquet au sein de l'application.
\end{document}