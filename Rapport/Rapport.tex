\documentclass[]{article}
\usepackage[utf8]{inputenc}
\usepackage{fullpage}
\usepackage{gensymb}
\usepackage{graphicx}
\usepackage{amsmath}

\begin{document}

\title{
{Rapport de projet}\\
\smallskip
{\Huge Go-back-N \& Congestion}\\
Réseau I\\
\author{Membres du groupe:\\
\textbf{SAHLI Yacine}\\
\textbf{HUYLENBROECK Florent}\\
}}

\date{Année Académique 2017-2018\\
Bachelier en Sciences Informatiques\\
\vspace{1cm}
Faculté des Sciences, Université de Mons}

\maketitle
%---------------------------------------------------------------------------------------------------------
\newpage
\section{Implémentation}
\subsection{L'application}
\subsection{Génération d'évènements aléatoires}
Problèmes simulés aléatoirement (voir "Paramètres modifiables" pour les probabilités de ceux-ci) et comportement de l'application lorsqu'il surviennent:
\begin{itemize}
\item \textbf{Ajout d'un court délai à certains ACK.}\\
Le thread est endormi pendant un nombre paramétrable de millisecondes avant l'envoi de l'ACK.
\item \textbf{Ajout d'un délai supérieur à la valeur du timeout à certains ACK.}\\
Le thread est endormi pendant un nombre non-paramétrable de millisecondes (correspondant au\\ TIMEOUT\_ DELAY dans \emph{reso.examples.gobackn.GoBackNSenderApp.java} avant l'envoi de l'ACK.
\item \textbf{Non-envoi de certains ACK.}\\
L'envoie de l'ACK correspondant est annulé.
\item \textbf{Perte de certains paquets.}\\
L'application agit comme si elle n'avait rien reçu.
\item \textbf{Certains paquets seront traités comme étant corrompus.}\\
L'application renvoie l'ACK correspondant au dernier paquet vérifié.
\end{itemize}
\subsection{Go-back-N}
\subsection{Contrôle de congestion}
\section{Utilisation de l'application}
\subsection{Lancement de l'application}
Pour démarrer l'application, lancer le fichier Main.java du package \emph{reso.examples.gobackn}  .
\subsection{Paramètres modifiables}
\begin{itemize}
\item \emph{reso.examples.gobackn.GoBackNReceiverApp.java}
\begin{itemize}
\item NUMBER\_ OF\_ EVENT et les variables commençant par PROB\_ permettent de modifier les probabilités que des évènements inattendus se produisent, selon la formule : \[\frac{\text{PROB\_ EVENT}}{\text{NUMBER\_ OF\_ EVENT}}\]\\
\emph{exemple} : Si l'on souhaite que $5\%$ des ACK ne soient pas envoyés,\\ PROB\_ ACK\_ NOT\_ SENT=5; et NUMBER\_ OF\_ EVENT=100;\\\\
La variable NUMBER\_ OF\_ EVENT sert principalement à augmenter/réduire toutes les probabilités en une seule fois.
\item SMALL\_ DELAY\_ RANGE\_ MIN et SMALL\_ DELAY\_ RANGE\_ MAX correspondent à l'intervalle dans lequel sera choisi les petits délais aléatoires avant d'envoyer certains ACK.\\
\end{itemize}
\item \emph{reso.examples.gobackn.GoBackNSenderApp.java}
\begin{itemize}
\item TIMEOUT\_ DELAY modifie le délai avant un timeout.
\item PACKET\_ SENT modifie le nombre de paquets à envoyer.
\end{itemize}
\item \emph{reso.examples.gobackn.Main.java}
\begin{itemize}
\item LINK\_ SIZE modifie la taille du lien entre les interfaces.
\item DEBIT modifie le débit.
\end{itemize}
\end{itemize}
\subsection{Comportement de l'application}
Une fois démarrée, une série de log s'afficheront dans la console, détaillant étape par étape l'échange de paquet au sein de l'application.\\
Un fichier \emph{log.txt} sera créé au même niveau que \emph{reso.examples.gobackn.Main.java} contenant les données nécessaire au plotting des résultats.
\end{document}